\documentclass[11pt]{article}   % 文档类型:文章,基准字号11pt


\setlength{\parindent}{0pt}
\usepackage{xltxtra}
\usepackage{hyperref}
\hypersetup{hidelinks}
\usepackage{url}
\urlstyle{tt}
\usepackage{xcolor}
\definecolor{CVBlue}{RGB}{23,110,191}
\usepackage{calc}
\usepackage{graphicx}
\usepackage{tikz}
\usetikzlibrary{calc}
\usepackage{fontspec}
\usepackage{xeCJK}
\usepackage{enumitem}
\CJKsetecglue{} %% 取消中文与数字之间的间隙


%% 主文档字体设置
\setmainfont[
    Path = fonts/Main/,
    Extension = .otf,
    BoldFont = texgyretermes-bold.otf, % 加粗字体
]{texgyretermes-regular.otf} % 正文字体

% 中文字体设置
\setCJKmainfont[
    Path = fonts/hansans/,
    Extension = .ttf,
    BoldFont = NotoSansSC-Bold.ttf, % 加粗字体
]{NotoSansSC-Regular.otf} % 正文字体


\usepackage{relsize}
\usepackage{xspace}

% 使用 fontawesome(部分图标)
\usepackage{fontawesome} 

% A4纸,上下左右边距
\usepackage[
    a4paper,
    left=1.2cm,
    right=1.2cm,
    top=1.5cm,
    bottom=1cm,
    nohead
]{geometry}

\renewcommand{\baselinestretch}{1.5} % 行间距设为1.5

\usepackage{titlesec}
\usepackage{enumitem}
\setlist{noitemsep} % 取消列表项间的额外间距
%\setlist{nosep} % 取消所有垂直间距
\setlist[itemize]{topsep=0.25em, leftmargin=*}
\setlist[enumerate]{topsep=0.25em, leftmargin=*}

% --- 用于控制【不同项目之间】的垂直距离 ---
\newlength{\interProjectSpacing}
\setlength{\interProjectSpacing}{0.9em} % <--- 在此调整项目之间的距离
\newcommand{\projectsep}{\vspace{\interProjectSpacing}}

% --- 用于控制【项目标题】与下方【项目描述】的距离 ---
\newlength{\intraProjectTitleSep}
\setlength{\intraProjectTitleSep}{0.4em} % <--- 在此调整标题和描述的距离
\newcommand{\titlebreak}{\\[\intraProjectTitleSep]}

% --- 用于控制【项目描述】与下方【要点列表】的距离 ---
\newlength{\intraProjectListTopSep}
\setlength{\intraProjectListTopSep}{0.2em} % <--- 在此调整描述和列表的距离

% =======================================================================


\titleformat{\section}         % 定制 \section 命令 
{\large\bfseries\raggedright} % 将 section 标题设置为大号、粗体且左对齐
{}{0em}                      % 可用于为所有 section 添加前缀(如“章节...”)
{}                           % 可用于在标题前插入代码
[{\color{CVBlue}\titlerule}]  % 在标题后插入一条横线
\titlespacing*{\section}{0cm}{*1.6}{*1.2}



\begin{document}
\pagenumbering{gobble}

%%%% 利用tikz来定位照片
\begin{tikzpicture}[remember picture, overlay] 
    \node[anchor = north east] at ($(current page.north east)+(-2cm,-0.5cm)$) {\includegraphics[height=3cm]{hjy.jpg}};
  \end{tikzpicture}%
  %%%% 利用tikz来定位学校Logo,这里只在第一页显示
  \begin{tikzpicture}[remember picture, overlay] 
    \node[anchor = north west] at ($(current page.north west)+(0.5cm,+1.0cm)$) {\includegraphics[height=6cm]{zju.png}};
  \end{tikzpicture}%
\centerline{\LARGE\bfseries{黄景圆}} 

\centerline{\normalsize{\faPhone\ 173-0098-9330 \quad \faEnvelopeO\ \href{mailto:22431119@zju.edu.cn}{22431119@zju.edu.cn}}} 

% \centerline{\normalsize{\faGithubSquare\ \href{https://github.com/maksymilan}{https://github.com/maksymilan} \quad \faRssSquare\ \href{https://maksymilan.github.io/}{https://maksymilan.github.io}}} 
    
\section{\makebox[\widthof{\faGraduationCap}][c]{\color{CVBlue}\faGraduationCap}\ 教育背景}    
\textbf{浙江大学} \hfill 2020.09 -- 2024.07\\[0.5em] % 标题和正文间加一点距离
电子科学与技术\quad 本科 \quad 信息与电子工程学院 
% \begin{itemize}[nosep]
%     \item 相关课程:《烂坑挖掘及基础》、《高级挖坑技巧》、《烂坑数理统计》
% \end{itemize}

\textbf{浙江大学} \hfill 2024.09 -- 至今\\[0.5em] % 标题和正文间加一点距离
新一代电子信息技术\quad 硕士 \quad 信息与电子工程学院 
% \begin{itemize}[nosep]
%     \item 相关课程:《烂坑挖掘及基础》、《高级挖坑技巧》、《烂坑数理统计》
% \end{itemize}

\section{\makebox[\widthof{\faUsers}][c]{\color{CVBlue}\faUsers}\ 项目经历}

% --- 第一个项目 ---
% 将标题行末尾的 \\ 替换为 \titlebreak 命令
\textbf{NFC虚拟仿真软硬件平台} \hfill 2024.12 -- 2025.10 \titlebreak
项目描述:针对 NFC 近场通信线圈耦合复杂、抗干扰设计难的痛点,搭建集\textbf{场路协同仿真}与\textbf{自主可控硬件验证}于一体的全链路闭环研究平台,实现从物理层电磁特性到协议层数据交互的端到端验证。
% 在 itemize 的选项中,使用 topsep=\intraProjectListTopSep 来控制上边距
\begin{itemize}[nosep, topsep=\intraProjectListTopSep]
    \item \textbf{场路联合仿真}:用 CST 建立多场景线圈模型并提取九宫格位置 S21 参数,在 ADS 搭建射频前端电路完成瞬态协同仿真,通过眼图分析验证弱耦合区域通信失效机理。
    \item \textbf{自主硬件设计}:基于 AVR 单片机(ATtiny84A/ATmega328P)实现 NFC Type 2 标签及 ISO14443-A 底层协议;基于 FPGA 用 Verilog 完成 13.56MHz 载波生成、ASK 调制及数字信号处理。
    \item \textbf{实测与模型优化}:搭建示波器测试平台对比仿真与实测波形,验证了仿真模型在时序和调制深度上的准确性,为NFC天线设计提供了可靠的虚拟验证环境。
\end{itemize}

% 使用 \projectsep 命令来分隔两个项目
\projectsep

% --- 第二个项目 ---
\textbf{AIGC驱动的超进化自动挖坑机 (Agent)} \hfill 2025.02 -- 至今 \titlebreak
项目描述:为探索心灵版规的边界,以及测试版主锁沉反应速度的极限,本项目基于\textbf{多智能体协作(MCP)与LLM}技术,开发了一款能\textbf{自主“挖坑”与“互坑”}的AI Agent。
\begin{itemize}[nosep, topsep=\intraProjectListTopSep]
    \item \textbf{并发与调度}:后端采用 \textbf{Go} 语言,充分利用 \textbf{Goroutine} 和 \textbf{Channel} 的高并发模型,模拟\textbf{用户并发在线},执行定时挖坑,确保挖坑行动的\textbf{隐蔽性}与\textbf{高效性}。
    \item \textbf{实时监控与响应}:运用 \textbf{WebSocket} 协议实现了对目标帖子的\textbf{实时监听与交互}。确保在烂坑被识破的第一时间,Agent能光速完成\textbf{“lktp”}或\textbf{反向钓鱼}操作,展现出极高的AI博弈能力。
    \item \textbf{可视化GUI}:前端使用 \textbf{React} 框架构建了\textbf{后台管理的可视化面板 (Admin Panel)}。可一键下达挖坑指令、动态调整Agent的\textbf{指令}、并实时监控数据。
\end{itemize}

\section{\makebox[\widthof{\faCogs}][c]{\color{CVBlue}\faCogs}\ 技术栈}
\begin{itemize}[nosep]
    \item \textbf{编程语言:} \textbf{Go}, Python, C++
    \item \textbf{开发工具:} SSH, Git, Vim, MakeFile,LaTex
    \item \textbf{操作系统:} Linux
\end{itemize}
\section{\makebox[\widthof{\faGraduationCap}][c]{\color{CVBlue}\faList}\ 获奖情况}
\begin{itemize}
    \item CC98年度用户 \hfill 2023.12
    
\end{itemize}
    
\section{\makebox[\widthof{\faInfo}][c]{\color{CVBlue}\faInfo}\ 其他}
\begin{itemize}[parsep=0.5ex]
    \item \textbf{技术博客:} \href{https://maksymilan.github.io/}{https://maksymilan.github.io/}
    \item \textbf{GitHub:} \href{https://github.com/maksymilan}{https://github.com/maksymilan} 
    \item \textbf{英语水平:} CET-4, CET-6
\end{itemize}
\end{document}